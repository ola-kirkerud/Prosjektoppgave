%===================================== CHAP 1 =================================

\chapter{Introduction}
In this chapter an introduction to the project is given. The motivations behind the project,
a description of the problem, the authors contribution to the problem and prevous work on this
field is presented. Lastly an outline of the report is given. 

\section{Motivation}
As the world moves into the direction of autonomous transportation, the development of
algorithms that ensures safe travel are of the utmost imporance. Research on 
collision avoidance is paramount to guarantee the safty of autonomous transportation.

The Norwegian Public Roads Administration wishes to both digitalize and autonomize the Norwegian
ferry system. As a response to this the Norwegian University of Science and Technology have
granted much attention to the research of autonomous seakeeping. Two small prototype ferries have
been build to work as a platform to do testing in a real world scenario.

Autonomous ferries can replace the need to build bridges over water, in both cities and rural areas.
Building bridges across the Norwegian fjords are an expensive endeavor that takes many years and 
have a big impact on the local enviroment. An autonomous ferry can in this scenario be deployed
significantly faster for a much lower cost than a brige. The building of the facilities for the ferries such as
docks will also have a much lower impact on the local enviroment. In the cities the demand of ferries are quite
different to those in more rural areas. With a growing population, the cities grow bigger and people
are spread out over a larger area. The wish for efficient, low cost and climate friendly alternatives to transportation
is bigger than ever. Electric autonomous ferries can transport passengers along the waterways with little enviromental impact
such as noice and pollutions. There is a considerable international market for these vessels and research on this topic will
lead to a speedier deployment. 


\section{Problem}

A description of the problem i am trying to solve. 

\section{Contributions?}

\section{Previous work}
- History on autonomous seakeeping
Waterways have been used as transportation by humans for as long as we know. From sailboats to large engine powered vessels.
The evolution of maritime navigation has been increadible, and the innovation keeps going. Right now most commercial vessels are controlled by a operator who
chooses the route and sets waypoints that the vessel should drive through. Choosing a safe and efficient route is not a simple task and require a experienced and skillfull operator,
and still accidents can occur. Research has therefor been done into giving the operator the tools for detecting potential collisions with other vessels or land. The operator
can then take evasive actions. Another great focus of research has been to take the operator out of the equation, making fully autonomous systems that can handle potential collisions
on it's own. These COLAV systems will need a high level of robustnes (noe mer...)




- Then come into colav systems and colreg
- Move into potential fields and how this has impacted reactive collison avoidance.


\section{Outline}
A description of what you will find in the different chapters. 

\cleardoublepage